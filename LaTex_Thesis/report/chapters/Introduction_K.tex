% !TeX root = ../main.tex
% Add the above to each chapter to make compiling the PDF easier in some editors.

\chapter{Introduction}\label{chapter:intro}
In recent decades, autonomous systems have been playing increasingly essential roles in industry, healthcare, and our daily life. Particularly, they help humans reduce the heavy workload and perform tasks effectively. Motivated by this, automation scientists and engineers are always seeking reliable and effective solutions for various systems to solve any specific tasks automatically. Many automation ideas have been already proposed and applied in the industry - automotive, manufacturing, etc. The applications produce great results and improve the quality of human life. Nevertheless, there are still many complicated tasks that require sophisticated solutions. For instance, tasks such as area sensing, surveillance, and rescue operations can not be solved by a individual system. This motivates researchers to design working strategies for Multi-Agent-Systems.\\
%\section{Multi-Agent-System}
%The terminology MAS refers to an autonomous system that consists of multiple agents with specific characteristics. The most significant difference between an MAS and a single-agent-system (SAS) is determined by their control method. In MASs, each agent may not have the total overview of the task, or the behavior of one agent may make impacts on the performances of another agents. Therefore, not every proposed control method for a SAS can be applied on a MAS without degradation in performance.\\
%The word agent refers to any kind of individual system with its own dynamics and characteristics, for example, wheeled mobile robots, unmanned aerial vehicles, or micro mouses.
\noindent The word \textit{Agent} refers to any kind of individual system with its own dynamics and characteristics, for example, wheeled mobile robots, unmanned aerial vehicles, or micro mouses. \\
\noindent The terminology \textit{Multi-Agent System - MAS} refers to an autonomous system that consists of multiple agents with specific characteristics. In comparison to a single-agent-system (SAS), an agent in a MAS just owns a limited amount of information and uses a decentralized control law during the operation. The total workload is divided into many agents so a MAS achieves higher performance than a SAS in many tasks. On the other hand, designing a control strategy for a MAS is challenging due to the property of distributed control method and the cooperation between many agents.\\
\section{Motivation}
In this thesis, we focus on the problem of coverage control executed by a MAS. According to the definition, coverage control implies a task, in which a group of many agents tries to cover the desired region for some specific purposes, such as exploration, data collection, or rescue operation. For instance, a group of aerial vehicle covers a predefined region for monitoring.
The problem is complicated because it requires agents to cooperate with each other to find a solution themselves. Furthermore, the limited communication, e.g due to the long-distance, does not allow an agent to obtain the information of all other agents. This also enhances the complexity of the task. The pioneering works in [2],[3] introduce the concept of coverage control by using a simple model to perform the task. [1],[3],[5],[6],[9] study the problem of coverage using models with higher complexity such as Unmanned Aerial Vehicles or WMR. 
Motivated by the application of MAS from the practical aspect, this thesis considers the problem of constraints related to WMR in the coverage control. For instance, a WMR must maintain a specific range in relation to the region to guarantee a reliable connection, this refers to the state constraint of the system. Furthermore, due to the physical restriction of the hardware components, the velocity of WMR is also limited. This refers to the input constraint. These constraints either exist due to the nature of the problem or they are intentionally imposed by humans to enhance the reliability and compatibility of the MAS. Indeed, constraints are ubiquitous in real situations so we are interested in this research direction.\\
\section{Contributions}
In general, the main theoretical contribution of this thesis is to study the above-mentioned constraints of WMR in coverage problems and propose a control method to handle the constraints. We note that there are some researches also considers constraints in coverage control. In [6],[7],[15],[16],[17],[18] the input saturation is considered for a group of agent operating coverage with different formulation. Respectively, [11],[20],[21] propose techniques to handle state constraints. The study in [1] proposes a switching control law to ensure an agent does not leave the coverage region while executing the task. This work is motivated to study the existence of a control law that can satisfy many constraints at the same time and still ensure the operational performance. Indeed, additionally considering a constraint is challenging and the existence of a solution is not always guaranteed. For instance, there might exist conflicts between the condition of constraints, so a controller that can handle input saturation is not able to ensure the state feasibility and vice versa. Moreover, due to the requirement that the heading velocity must be constant, it implies that a WMR is underactuated. The highlight of this thesis is the design of a controller that can find a compromise between many constraints and is therefore applicable in real situations. Furthermore, the theoretical and analytical derivation used to handle various constraints at the same time can be useful for many related researches. 
\section{Organization}
%The thesis is organized as follows\\
Chapter 2 introduces the necessary preliminaries for the whole thesis. Chapter 3 studies the problems of the constraints and theoretically proposes a control law to overcome these problems. In chapter 4, we present the simulation platform that we create to simulate a group of WMR executing the coverage task. The performance of the proposed controller is also evaluated in this chapter. Chapter 5 describes the limitations of the proposed control method. In chapter 6, we conclude the thesis and propose the future work.







