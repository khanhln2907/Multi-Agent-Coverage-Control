\chapter{Technical Approach}

\section{Design of your solution}

Having explained the problem, and what others have done in similar situations, now explain your approach. Again, give a general overview of your design first, and then go into detail. The important part here is the concept of your work, not the actual implementation! Make sure that the document (particularly a thesis) is self-contained: It should be possible for a reader familiar with the general area to understand your design. Again, be forthright about the limitations of your design. Also, make sure you justify any shortcuts/limitations convincingly.

\section{Implementation}

In many (not all cases) there is a clear difference between the general approach (design) and its implementation in your particular circumstances. The design may be more general than what you can do given time and resources. Or you have developed a general design, and are now implementing a prototype on particular hardware. Give all required details. It should be possible to understand all this without referring to the source code. 

This will, in general, include extracts of actual algorithms and hardware components used. Don't list pages of C code, an electronic copy of the source will accompany the submission and should be available to the marker, so there's no point in killing extra trees to put it into the report. Source code, if included at all, goes into the appendix and not the main document.

Make sure you describe your implementation in enough detail. Someone who has nothing else but your thesis report to go by should be able to repeat your work, and arrive at essentially the same implementation. Reproducibility is an important component of scientific work. Also, clearly state the limitations of your implementation, and justify them.


